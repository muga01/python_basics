\documentclass[a4paper]{article}
\usepackage{varioref}
%%\usepackage{placeins}
\usepackage{float}
\usepackage{multirow,tabularx}
%%Edycja czcionki
%\usepackage{polski}
\usepackage[utf8]{inputenc}
\usepackage[T1]{fontenc}
\usepackage{pdfpages}

\usepackage{listings}
\usepackage{color} %red, green, blue, yellow, cyan, magenta, black, white
\definecolor{mygreen}{RGB}{28,172,0} % color values Red, Green, Blue
\definecolor{mylilas}{RGB}{170,55,241}

\newcommand\tab[1][1cm]{\hspace*{#1}}

%% Sets page size and margins
\usepackage[a4paper,top=2cm,bottom=1cm,left=2cm,right=2cm,marginparwidth=1.75cm]{geometry}

%% Useful packages
\usepackage{amsmath}
\usepackage{graphicx}
\usepackage[colorinlistoftodos]{todonotes}
\usepackage[colorlinks=true, allcolors=blue]{hyperref}


%%Początek dokumentu
\begin{document}

\lstset{language=Matlab,%
    %basicstyle=\color{red},
    breaklines=true,%
    morekeywords={matlab2tikz},
    keywordstyle=\color{blue},%
    morekeywords=[2]{1}, keywordstyle=[2]{\color{black}},
    identifierstyle=\color{black},%
    stringstyle=\color{mylilas},
    commentstyle=\color{mygreen},%
    showstringspaces=false,%without this there will be a symbol in the places where there is a space
    numbers=left,%
    numberstyle={\tiny \color{black}},% size of the numbers
    numbersep=9pt, % this defines how far the numbers are from the text
    emph=[1]{for,end,break},emphstyle=[1]\color{red}, %some words to emphasise
    %emph=[2]{word1,word2}, emphstyle=[2]{style},    
}

%%Początek stony tytułowej
\begin{titlepage}
    \centering
	\begin{figure}[H]
	\centering
	\includegraphics[width=0.3\textwidth]{LOGO.jpg}
	\end{figure}
    \vspace*{0.5cm}
    \huge\bfseries
    Laboratory report \\Markov Models \\Laboratory 2 \\
    \vspace*{0.1cm}
    \vspace*{0.5cm}
    \vspace*{0.5cm}
    \vspace*{1cm}
    \large Radosław Milenkowicz\\

    \vspace*{\fill}
    \large Gliwice 13.10.2021 \\
\end{titlepage}
\newpage


\section{Task 1}
\subsection{Analysing graphs.}
\begin{enumerate}
    \item As could be predicted based on state matrix and observed on graphs of state transitions: the  state prefers to stay in state 3 and barely any time jumps to state 1 or state 2. State 1 has really low chance to stay in the state 1 or to jump state 2, State 2 has "alike" state transition probabilities, although bit larger.
    \item Starting from state 2 follows the same fate as starting from state 1 since transition matrix is the same and the probability of transitioning to state 3 still dominates the state transitions.
\end{enumerate}
\subsection{Data Visualisation for state no.1 = 1}

\begin{figure}[H]
\centering
\includegraphics[width=1\textwidth]{T1_a_Figure_1.png}
\caption{\label{fig:table}First simulation.}
\end{figure}
\begin{figure}[H]
\centering
\includegraphics[width=1\textwidth]{T1_a_Figure_2.png}
\caption{\label{fig:table}Second simulation.}
\end{figure}

\begin{figure}[H]
\centering
\includegraphics[width=1\textwidth]{T1_a_Figure_3.png}
\caption{\label{fig:table}Third simulation.}
\end{figure}

\begin{figure}[H]
\centering
\includegraphics[width=1\textwidth]{T1_a_Figure_4.png}
\caption{\label{fig:table}Fourth simulation.}
\end{figure}

\begin{figure}[H]
\centering
\includegraphics[width=1\textwidth]{T1_a_Figure_5.png}
\caption{\label{fig:table}Fifth simulation.}
\end{figure}

\subsection{Data Visualisation state for no.1 = 2}

\begin{figure}[H]
\centering
\includegraphics[width=1\textwidth]{T1_b_Figure_1.png}
\caption{\label{fig:table}First simulation.}
\end{figure}
\begin{figure}[H]
\centering
\includegraphics[width=1\textwidth]{T1_b_Figure_2.png}
\caption{\label{fig:table}Second simulation.}
\end{figure}

\begin{figure}[H]
\centering
\includegraphics[width=1\textwidth]{T1_b_Figure_3.png}
\caption{\label{fig:table}Third simulation.}
\end{figure}

\begin{figure}[H]
\centering
\includegraphics[width=1\textwidth]{T1_b_Figure_4.png}
\caption{\label{fig:table}Fourth simulation.}
\end{figure}

\begin{figure}[H]
\centering
\includegraphics[width=1\textwidth]{T1_b_Figure_5.png}
\caption{\label{fig:table}Fifth simulation.}
\end{figure}

\newpage 
\section{Task 2}
\subsection{Task 2: A}

Inital state: state = [[7400000 ,  8370000 ,  7900000]] \\
Transition state matrix: \\
1:[0.03, 0.07, 0.9], \\
2:[0.15, 0.15, 0.70], \\
3:[0.25, 0.20, 0.55]

\begin{figure}[H]
\centering
\includegraphics[width=1\textwidth]{T2_a_Figure_1.png}
\caption{\label{fig:table}Twelve month simulation.}
\end{figure}

\begin{figure}[H]
\centering
\includegraphics[width=1\textwidth]{T2_a_Figure_2.png}
\caption{\label{fig:table}First month.}
\end{figure}

\begin{figure}[H]
\centering
\includegraphics[width=1\textwidth]{T2_a_Figure_3.png}
\caption{\label{fig:table}Sixth Month.}
\end{figure}

\begin{figure}[H]
\centering
\includegraphics[width=1\textwidth]{T2_a_Figure_4.png}
\caption{\label{fig:table}After a year.}
\end{figure}
\newpage
\section{Task 2b)}
Inital state: state = [[7400000 ,  8370000 ,  7900000]] \\
Transition state matrix: \\
I must refer to my appendix A: state\_transition\_b(\_b1 \_b2 \_b3)
are described there.
\subsection{Simulation}
\begin{figure}[H]
\centering
\includegraphics[width=1\textwidth]{T2_b_Figure_1.png}
\caption{\label{fig:table}Twelve month simulation.}
\end{figure}

\subsection{Bar charts}
\begin{figure}[H]
\centering
\includegraphics[width=1\textwidth]{T2_a_Figure_2.png}
\caption{\label{fig:table}First month.}
\end{figure}

\begin{figure}[H]
\centering
\includegraphics[width=1\textwidth]{T2_b_Figure_2.png}
\caption{\label{fig:table}After 6 months.}
\end{figure}

\begin{figure}[H]
\centering
\includegraphics[width=1\textwidth]{T2_b_Figure_3.png}
\caption{\label{fig:table}After a year.}
\end{figure}
\newpage 
\section{Appendix A: Code listening}

\begin{lstlisting}

def state_transition_row_a(state):
    transition_row_a={
        1:[0.03, 0.07, 0.9],
        2:[0.15, 0.15, 0.70],
        3:[0.25, 0.20, 0.55]
    }
    return transition_row_a.get(state,"Invalid state")

def state_transition_row_b(state):
    transition_row_b={
        1:[0.83, 0.05, 0.12],
        2:[0.12, 0.72, 0.16],
        3:[0.06, 0.03, 0.91]
    }
    return transition_row_b.get(state,"Invalid state")

def state_transition_row_b_1(state):
    transition_row_b={
        1:[0.88, 0.025, 0.095],
        2:[0.19, 0.65, 0.16],
        3:[0.08, 0.02, 0.90]
    }
    return transition_row_b.get(state,"Invalid state")

def state_transition_row_b_2(state):
    transition_row_b={
        1:[0.915, 0.025, 0.06],
        2:[0.195, 0.66, 0.145],
        3:[0.1, 0.04, 0.86]
    }
    return transition_row_b.get(state,"Invalid state")

def state_transition_row_b_3(state):
    transition_row_b={
        1:[0.895, 0.005, 0.04],
        2:[0.175, 0.64, 0.125],
        3:[0.095, 0.035, 0.855]
    }
    return transition_row_b.get(state,"Invalid state")

def fetch_state_transition_row_chaning(months,state):
    transition_matrixes={
    1:state_transition_row_b(state),
    2:state_transition_row_b(state),
    3:state_transition_row_b_1(state),
    4: state_transition_row_b_1(state),
    5: state_transition_row_b_1(state),
    6: state_transition_row_b_1(state),
    7: state_transition_row_b_1(state),
    8: state_transition_row_b_2(state),
    9: state_transition_row_b_2(state),
    10: state_transition_row_b_3(state),
    11: state_transition_row_b_3(state),
    12: state_transition_row_b_3(state),
    }
    return transition_matrixes.get(months, "Invalid state")


if __name__ == '__main__':

    Task1:
    for y in range(5):
        #Array of states first state is beginning state, will consist of 51 states in total(for 50 transitions)
        state_list = [2]

        #state_row = state_transition_row(3)
        for x in range(50):
            # Take the last state known from the state list
            transition_row = state_transition_row_a(state_list[-1])
            new_state = np.random.multinomial(1, transition_row)
            # Cast type of numpy array to a pythonic list so that index method works
            new_state = new_state.tolist()
            # find first element that has value 1 and add 1 so that it matches state "number"
            ind = new_state.index(1) + 1
            # Add the state to the state list
            state_list.append(ind)
            pass


        fig, ax = plt.subplots()
        ax.plot(state_list)
        locs, labels = plt.yticks()  # Get the current locations and labels.
        plt.yticks(np.arange(1, 4, step=1))
        locs, labels = plt.xticks()  # Get the current locations and labels.
        plt.xticks(np.arange(1, 51, step=1))
        ax.set(xlabel='iteration(n)', ylabel='state',
               title='Task1 plot')
        ax.grid()
        plt.show()


    #Task 2a):
    state = [[7400000,8370000,7900000]]
    for x in range(12):
        #Fetch the last month:
        current_state = state[-1]

        #Grab "populations" from current month:
        danone_population = current_state[0]
        mlekovita_population = current_state[1]
        zott_population = current_state[2]

        # Fetch data that defines how do they move around:
        transition_row_Danone = state_transition_row_b(1)
        transition_row_Mlekovita = state_transition_row_b(2)
        transition_row_Zott = state_transition_row_b(3)

        # Let them choose what they want...:
        new_state_d = np.random.multinomial(danone_population, transition_row_Danone)
        new_state_m = np.random.multinomial(mlekovita_population, transition_row_Mlekovita)
        new_state_z = np.random.multinomial(zott_population, transition_row_Zott)

        #Define new state:
        new_state = new_state_z + new_state_d + new_state_m
        # Add a month to the list:
        state.append(new_state)

    danone,mlekovita,zott = zip(*state)

    fig, ax = plt.subplots()
    ax.plot(danone,'b')
    ax.plot(mlekovita,'r')
    ax.plot(zott,'g')
    locs, labels = plt.xticks()  # Get the current locations and labels.
    plt.xticks(np.arange(0, 13, step=1))
    plt.yticks(np.arange(2000000, 15000000, step=1000000))
    #plt.yticks(np.arange(1, 12, step=1))
    ax.set(xlabel='iteration(i)', ylabel='number of clients(n)',
           title='Task2a) plot')
    plt.legend(["danone", "mlekovita","zott"], loc ="upper left")
    ax.grid()
    plt.show()


    Companies = ('Danone', 'Mlekovita', 'Zott')
    y_pos = np.arange(len(Companies))
    clients = state[6]
    plt.bar(y_pos, clients, align='center', alpha=1)
    plt.xticks(y_pos, Companies)
    plt.ylabel('Clients')
    plt.title('Clients after half a year')

    plt.show()

    # Task 2b):
    state = [[7400000, 8370000, 7900000]]
    for x in range(12):
        # Fetch the last month:
        current_state = state[-1]

        # Grab "populations" from current month:
        danone_population = current_state[0]
        mlekovita_population = current_state[1]
        zott_population = current_state[2]

        # Fetch data that defines how do they move around:
        transition_row_Danone = fetch_state_transition_row_chaning(len(state),1)
        transition_row_Mlekovita = fetch_state_transition_row_chaning(len(state),2)
        transition_row_Zott = fetch_state_transition_row_chaning(len(state),3)

        # Let them choose what they want...:
        new_state_d = np.random.multinomial(danone_population, transition_row_Danone)
        new_state_m = np.random.multinomial(mlekovita_population, transition_row_Mlekovita)
        new_state_z = np.random.multinomial(zott_population, transition_row_Zott)

        # Define new state:
        new_state = new_state_z + new_state_d + new_state_m
        # Add a month to the list:
        state.append(new_state)

    danone, mlekovita, zott = zip(*state)

    fig, ax = plt.subplots()
    ax.plot(danone,'b')
    ax.plot(mlekovita,'r')
    ax.plot(zott,'g')
    locs, labels = plt.xticks()  # Get the current locations and labels.
    plt.xticks(np.arange(0, 13, step=1))
    plt.yticks(np.arange(2000000, 15000000, step=1000000))
    #plt.yticks(np.arange(1, 12, step=1))
    ax.set(xlabel='iteration(i)', ylabel='number of clients(n)',
           title='Task2b) plot')
    plt.legend(["danone", "mlekovita","zott"], loc ="upper left")
    ax.grid()
    plt.show()

    Companies = ('Danone', 'Mlekovita', 'Zott')
    y_pos = np.arange(len(Companies))
    clients = state[0]
    plt.bar(y_pos, clients, align='center', alpha=1, color=["blue","red","green"])
    plt.xticks(y_pos, Companies)
    plt.ylabel('Clients')
    plt.title('Clients in the beginning')

    plt.show()
\end{lstlisting}

\end{document}
%